%% http://www.tex.ac.uk/FAQ-keyval.html <= key-value

\documentclass{article}

%%% Les packages nécessaires.
\usepackage{keyval}
\usepackage{tikz}
\usetikzlibrary{decorations.markings,arrows}

\makeatletter
\newcommand{\DrawRecursiveSequence}[4][]{%
    %
    % #1 => Options. The default are [color=black,arrowscale=0.4,sep=0.3pt,dotsize=0.7pt,seqname=u,first=0].
	% #2 => Function macro with a definition like that: \def\f(#1){...} where #1 is the variable.
	% #3 => The first value of the sequence.
	% #4 => Number of elements to draw.
	%
	\def\DRS@color{black}%
	\define@key{DRS@options}{color}{%
	    \def\DRS@color{##1}%
	}%
	\def\DRS@arrowscale{0.4}%
	\define@key{DRS@options}{arrowscale}{%
	    \def\DRS@arrowscale{##1}%
	}%
	\def\DRS@sep{3pt}%
	\define@key{DRS@options}{sep}{%
	    \def\DRS@sep{##1}%
	}%
	\def\DRS@dotsize{0.7pt}%
	\define@key{DRS@options}{dotsize}{%
	    \def\DRS@dotsize{##1}%
	}%
	\def\DRS@seqname{u}%
	\define@key{DRS@options}{seqname}{%
	    \def\DRS@seqname{##1}%
	}%
	\def\DRS@first{0}%
	\define@key{DRS@options}{first}{%
	    \def\DRS@first{##1}%
	}%

	\setkeys{DRS@options}{#1}
	%
	\pgfmathsetmacro{\DRS@un}{#3}%
	\pgfmathsetmacro{\DRS@max}{#4 - 1}%
	\pgfmathsetmacro{\DRS@index}{\DRS@first}%
	%
    \begin{scope}[decoration={markings, mark=at position 0.6 with {\arrow[scale=\DRS@arrowscale]{triangle 45}}}] 

        %\clip (\xmin,\ymin) rectangle (\xmax,\ymax);

        \pgfmathsetmacro{\DRS@unPU}{#2(\DRS@un)}
        \draw[dashed,color=\DRS@color] (\DRS@un,0) -- (\DRS@un, \DRS@unPU)
            node[inner sep=\DRS@sep,anchor=base,below] at (\DRS@un,0) {$\DRS@seqname_{\pgfmathprintnumber[int trunc]{\DRS@index}}$};
        \draw (\DRS@un,0) node[color=\DRS@color,circle,fill,inner sep=\DRS@dotsize] {};
        \draw (\DRS@un,\DRS@unPU) node[color=\DRS@color,circle,fill,inner sep=\DRS@dotsize] {};

        \foreach \n in {1, ..., \DRS@max} {
            \pgfmathsetmacro{\DRS@unPU}{#2(\DRS@un)}
            \pgfmathsetmacro{\DRS@unPD}{#2(\DRS@unPU)}
            \pgfmathsetmacro{\DRS@index}{\DRS@index + 1}
            \draw[color=\DRS@color,postaction={decorate}] (\DRS@un,\DRS@unPU) -- (\DRS@unPU,\DRS@unPU);
            \draw[color=\DRS@color,postaction={decorate}] (\DRS@unPU,\DRS@unPU) -- (\DRS@unPU,\DRS@unPD);
            %\draw (\DRS@unPU,\DRS@unPU) node[circle,fill,inner sep=0.7pt] {};
            \draw (\DRS@unPU,\DRS@unPD) node[color=\DRS@color,circle,fill,inner sep=\DRS@dotsize] {};
            \draw (\DRS@un,\DRS@unPU) node[color=\DRS@color,circle,fill,inner sep=\DRS@dotsize] {};
            \draw[dashed,color=\DRS@color] (\DRS@unPU,0) -- (\DRS@unPU, \DRS@unPU)
                node[inner sep=\DRS@sep,anchor=base,below] at (\DRS@unPU,0) {$\DRS@seqname_{\pgfmathprintnumber[int trunc]{\DRS@index}}$};
            \draw (\DRS@unPU,0) node[color=\DRS@color,circle,fill,inner sep=\DRS@dotsize] {};
            \global\let\DRS@un=\DRS@unPU
            \global\let\DRS@index=\DRS@index
        }
    \end{scope}
}
\makeatother
%%%
%%% Fin de la définition de la macro « magique ».
%%%

\begin{document}

\begin{center}

    \newcommand{\scale}{1.25}
    
    \begin{tikzpicture}[scale=\scale]

        %%% Ça c'est la partie pour tracer les axes, les graduations etc.
        %%% Ce n'est pas le point qui t'intéresse ici.
        \newcommand{\xmin}{0}
        \newcommand{\xmax}{5}
        \newcommand{\ymin}{0}
        \newcommand{\ymax}{5}

        \draw[->] (\xmin,0) -- (\xmax,0) node[right] {$x$};
        \draw[->] (0,\ymin) -- (0,\ymax) node[above] {$y$};
        \pgfmathsetmacro{\ticsize}{0.05/\scale}
        \node[below left] at (-\ticsize,-\ticsize) {$0$};
        \foreach \x in {1,...,4} {
            \draw (\x,\ticsize) -- (\x,-\ticsize); %node[below] {$\x$};
        }
        \foreach \y in {1,...,4} {
            \draw (\ticsize,\y) -- (-\ticsize,\y); %node[left] {$\y$};
        }
        %%% Fin de la tambouille pour tracer les axes etc.
        %%% Le coeur du sujet, c'est la suite.



        %%% Définition de la fonction f. Attention, la syntaxe est un peu
        %%% spéciale, c'est :
        %%%
        %%%		\def\lafonction(#1){... définition de la fonction où #1 représente la variable ...}
        %%%
        %%% Ici la fonction s'appelle juste \f mais on pourrait choisir un autre nom.
        \def\f(#1){(#1+3)/4)^2+1}

        %%% Tracé de la fonction f.
        \draw[very thick,color=blue,smooth] plot[domain=\xmin:\xmax,samples=100] ({\x},{\f(\x)});
        
        %%% Tracé de la droite y = x
        \draw[very thick,color=green!75!black] plot[domain=\xmin:\xmax,samples=2] ({\x},{\x});

		%%% Dans l'ordre, les arguments sont:
		%%%
		%%% - La fonction (\f ici)
		%%% - La valeur du premier terme de la suite (0.5 ici).
		%%% - Le nombre de termes de la suite qu'on veut tracer.
        %\DrawRecursiveSequence[color=red,arrowscale=0.5,sep=3pt,dotsize=0.5pt,seqname=v,first=2]{\f}{0.5}{3}
        \DrawRecursiveSequence{\f}{0.5}{3}

    \end{tikzpicture}
    
\end{center}



\end{document}

